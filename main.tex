%페이지 세팅
\documentclass[9.5pt,a4paper]{article}
\setlength{\parindent}{0em}                  %DISTANCIA SANGRÍA
\setlength{\parskip}{0.5em}                  %DISTANCIA ENTRE PÁRRAFOS
\textwidth 6.5in
\textheight 9.in
\oddsidemargin 0in
\headheight 0in

\usepackage{amsmath}
\usepackage{tcolorbox}
\usepackage{amssymb}
\usepackage{amsthm}
\usepackage{lastpage}
\usepackage{fancyhdr}
\usepackage{accents}
\usepackage{setup}
\usepackage{import}
\usepackage{fancyhdr}
\usepackage{layouts}
\addtolength{\voffset}{0mm}
\addtolength{\textheight}{0mm}

\usemintedstyle{manni}

\usepackage{xcolor}
\usepackage{mdframed}
\usepackage[shortlabels]{enumitem}
\usepackage{indentfirst}
\usepackage{hyperref}
\renewcommand{\thesubsection}{\thesection.\alph{subsection}}
\linespread{1.15}
%%%%%%%%%%%%%%%%%%%%%%%%%%%%%%%%%%%%%%%%%%%%%%%%%%%%%%%%%%%%%%%%%%%%%%%%%%
    % 매주 리포트를 작성할때 이 부분을 수정하면 보고서 전체가 수정된다. 
%%%%%%%%%%%%%%%%%%%%%%%%%%%%%%%%%%%%%%%%%%%%%%%%%%%%%%%%%%%%%%%%%%%%%%%%%%
% 작성하는 주차 
\newcommand{\numnum}{03}
% 이번 실험의 부제목 
% \newcommand{\subtitle}{Install Unity, SMPL Application and OBJ format}
\newcommand{\subtitle}{OOP Basic}

\renewcommand{\listingscaption}{\footnotesize \textbf{python code}}
%-------------------------------------------------------------------------
%--> 떠다니는 객체 사용하는 부분 
\usepackage{wrapfig}
\newenvironment{problem}[2][Problem]  
    { \begin{mdframed}[backgroundcolor=gray!20] \textbf{#1 #2} \\}
    {  \end{mdframed}}
%-------------------------------------------------------------------------
\begin{document}
\pagestyle{fancy}
\fancyhf{}
    \rhead{\small 2016142096 조윤신}
    \lhead{\textsc{Homework \numnum \quad\subtitle}}
\cfoot{\thepage}
\renewcommand\headrulewidth{0.3mm}
%\renewcommand\footrulewidth{0.3mm}
\thispagestyle{plain}
\begin{flushleft}
\textsc{School of Electrical and Electronic of Enginnering} \\
\textsc{eee 3545-01 : Application Programming}\\[0.1cm]
\small{\textsc{ 2016142096 \textbf{조윤신}}}\\
\end{flushleft}
%-------------------------------------------------------------------------
\begin{flushright}\vspace{-25mm}
    \includegraphics[height=3cm]{image/1.jpg}
    \vspace{5mm}
\end{flushright}
    \begin{center}\vspace{-2cm}
        \textbf{\huge HomeWork \numnum}\\ 
        \vspace{0.6mm}
        {\large \subtitle}\\           
    \end{center}
\vspace{-5mm}
\rule{\linewidth}{0.4mm}
\vspace{-5mm}
% % main documents
%     \begin{mdframed}[backgroundcolor = gray!20] 
%         \vspace{-5mm}
%         \begin{center}
%             \subsubsection*{\large Homework Notice}
%                 \vspace{-2mm}
%                 \begin{itemize}
%                     \item {\small 
%                     Problem 1에 해당하는 태양, 지구 그리고 달의 자전과 공전을 담은 씬과 Problem 3에 해당하는 
%                     main camera의 display의 결과물을 각각 스크린 캡처했습니다.}
%                     \vspace{-2mm}
%                     \item {\small 비디오 파일은 유튜브에 업로드 하였습니다. 링크는 
%                     아래에서 확인할 수 있습니다.}
%                     \vspace{-2mm}
%                     \begin{itemize}
%                         \item \href{https://youtu.be/RvKIJbZvhXQ}
%                 		{HomeWork 02 - Problem 1 : Recorded Video Youtube Link}
%                 		\item \href{https://youtu.be/ssXaoGTbSxE}
%                 		{HomeWork 02 - Problem 3 : Recorded Video Youtube Link} 
%                     \end{itemize}
%                     % \item  {\small 
%                     % Pynq 보드가 어떻게 Audio를 출력하는지 이해하기
%                     % }
%                 \end{itemize}
%         \end{center}
%     \end{mdframed}
    
% 모든 프로그래밍 편집기 (IDE)에서 원도우 기준으로 '컨트롤 + /'를 누르면 
% 커서가 활성화된 line의 주석처리를 해제 지정할 수 있다. 
%%%%%%%%%%%%%%%%%%%%%%%%%%%%%%%%%%%%%%%%%%%%%%%%%%%%%%%%%%%%%%%%%%%%%%%%%%
                    % Report 1    Due : 09/14   개인
%%%%%%%%%%%%%%%%%%%%%%%%%%%%%%%%%%%%%%%%%%%%%%%%%%%%%%%%%%%%%%%%%%%%%%%%%%
% \import{./TEX/Report_01}{01}
% \import{./TEX/Report_01}{02}
%-------------------------------------------------------------------------
%%%%%%%%%%%%%%%%%%%%%%%%%%%%%%%%%%%%%%%%%%%%%%%%%%%%%%%%%%%%%%%%%%%%%%%%%%
                    % Report 2    Due : 09/14
%%%%%%%%%%%%%%%%%%%%%%%%%%%%%%%%%%%%%%%%%%%%%%%%%%%%%%%%%%%%%%%%%%%%%%%%%%
% \import{./TEX/Report_02}{01}
%-------------------------------------------------------------------------
%%%%%%%%%%%%%%%%%%%%%%%%%%%%%%%%%%%%%%%%%%%%%%%%%%%%%%%%%%%%%%%%%%%%%%%%%%
                    % Report 2    Due : 11/16
%%%%%%%%%%%%%%%%%%%%%%%%%%%%%%%%%%%%%%%%%%%%%%%%%%%%%%%%%%%%%%%%%%%%%%%%%%
\import{./TEX/Report_03}{01}
%-------------------------------------------------------------------------
\end{document}